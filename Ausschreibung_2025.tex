\documentclass[12pt]{article}
\usepackage[a4paper]{geometry}
\usepackage{xskak,chessboard}
\usepackage[ngerman]{babel}
\usepackage{fancyhdr,hyperref,multicol,caption,float}

\setlength{\headheight}{15pt}



\pagestyle{fancy}
\fancyhf{}
\lhead{Wahlfachwoche 2025}
\rhead{Schachwoche: Turniere und Varianten}
%\cfoot{Seite \thepage}
%\lfoot{Fa, DC}

%\title{Schach 1.0}
%\author{Daniel Fasnacht, Oliver De Capitani}
%\subtitle{Wahlfachwoche 2019}
%\date{24.-28. Juni 2019}

\parindent=0mm


\renewcommand{\arraystretch}{1.2}

\begin{document}
\setchessboard{showmover=false}


 
        \Large
        \textbf{Schachwoche: Turniere und Varianten}
 \vspace*{0.3cm}
 
         \Large
Projekt Wahlfachwoche 2025 (23.Juni - 27. Juni 2025) 
 \normalsize
 
 \vfill
 
        \textbf{Veranstaltende Lehrperson(en):} Oliver De Capitani (DC), Simeon Jackman (Jc)
 
        \vfill
        
\textbf{Ziel des Kurses:}

Spielen Sie gerne Schach und wollen Fortschritte in Ihrem Spiel machen? Dann sind Sie in der Schachwoche richtig!

Wir beschäftigen uns eine Woche lang auf vielfältigste Weise mit dem berühmtesten aller Brettspiele: Wir studieren die verschiedenen Phasen des Spiels von der Eröffnung bis zum Endspiel. Wir lösen Schachrätsel und führen ein Blitzturnier durch. Auch eine Simultanpartie gegen einen Vereinsspieler steht auf dem Programm, sowie ein Ausflug in einen Park, wo wir Gartenschach spielen werden.	

Egal ob Anfänger oder Fortgeschritten: Wenn Sie Freude haben, sich eine Woche lang intensiv mit Schach zu beschäftigen und Ihr Niveau steigern wollen, dann ist die Schachwoche genau das richtige für Sie.

\begin{multicols}{2}
\begin{figure}[H]
\centering
\chessboard[smallboard,
setfen=4r3/2k2p1p/B6b/P6p/2bP4/8/1P4PP/R2K3R b - - 2 24,
%arrow=to,linewidth=0.2ex,
%pgfstyle=straightmove,
%shortenstart=0.4em,
%color=red!80,
%markmoves={c8-f5}
]
\caption{Einfaches Rätsel: \\ Schwarz ist am Zug und setzt Weiss im nächsten Zug Matt.}
\end{figure}

\begin{figure}[H]
\centering
\chessboard[smallboard,
setfen=r2qk1r1/ppp1bp2/3p1p1p/3Np2Q/2B1P1b1/3P4/PPP2PPP/2KR3R w q - 6 12,
%arrow=to,linewidth=0.2ex,
%pgfstyle=straightmove,
%shortenstart=0.4em,
%color=red!80,
%markmoves={f4-f7},
]
\caption{Mittleres Rätsel: \\ Weiss kann Schwarz in zwei Zügen matt setzen.}
\end{figure}
\end{multicols}
 
 		\vfill
 		
 		\textbf{Minimalanforderungen (fachlich):} Sie kennen die Regeln des Schachspiels und können problemlos eine Partie Schach spielen. Wichtig: Es kann {\bf keine} Einführung in die Grundregeln angeboten werden. 
\vfill
\textbf{Maximale Teilnehmerzahl:} 16

\vfill
\textbf{Kosten pro SchülerIn:}	max. CHF10


\vfill

\textbf{Zimmer- und PC-Wünsche:} F0 und G0
                 






\end{document}